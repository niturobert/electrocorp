\chapter{System Administration}

\section*{3.1 \hspace{1cm} Docker Compose}
Docker Compose is a tool for defining and running multi-container Docker applications. It uses YAML files to configure the application's services and performs the creation and start-up process of all the containers with a single command. The docker-compose CLI utility allows users to run commands on multiple containers at once, for example, building images, scaling containers, running containers that were stopped, and more. Commands related to image manipulation, or user-interactive options, are not relevant in Docker Compose because they address one container. The docker-compose.yml file is used to define an application's services and includes various configuration options. For example, the build option defines configuration options such as the Dockerfile path, the command option allows one to override default Docker commands, and more. The first public beta version of Docker Compose (version 0.0.1) was released on December 21, 2013. The first production-ready version (1.0) was made available on October 16, 2014.

\section*{3.2 \hspace{1cm} Linux}
Linux is a family of open-source Unix-like operating systems based on the Linux kernel, an operating system kernel first released on September 17, 1991, by Linus Torvalds.Linux is typically packaged in a Linux distribution. \\

Distributions include the Linux kernel and supporting system software and libraries, many of which are provided by the GNU Project. Many Linux distributions use the word "Linux" in their name, but the Free Software Foundation uses the name "GNU/Linux" to emphasize the importance of GNU software, causing some controversy.

\section*{3.3 \hspace{1cm} iptables}
iptables is a user-space utility program that allows a system administrator to configure the IP packet filter rules of the Linux kernel firewall, implemented as different Netfilter modules. The filters are organized in different tables, which contain chains of rules for how to treat network traffic packets. Different kernel modules and programs are currently used for different protocols; iptables applies to IPv4, ip6tables to IPv6, arptables to ARP, and ebtables to Ethernet frames. \\

iptables requires elevated privileges to operate and must be executed by user root, otherwise it fails to function. On most Linux systems, iptables is installed as /usr/sbin/iptables and documented in its man pages, which can be opened using man iptables when installed. It may also be found in /sbin/iptables, but since iptables is more like a service rather than an "essential binary", the preferred location remains /usr/sbin.

