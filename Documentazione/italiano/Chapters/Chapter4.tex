\chapter{Amministrazione di Sistema}

\section*{3.1 \hspace{1cm} Docker Compose}
Docker Compose è uno strumento per la definizione e l'esecuzione di applicazioni Docker multi-container. Utilizza i file YAML per configurare i servizi dell'applicazione ed esegue il processo di creazione e avvio di tutti i contenitori con un unico comando. L'utilità CLIedocker-compose consente agli utenti di eseguire comandi su più contenitori contemporaneamente, ad esempio, la creazione di immagini, il ridimensionamento di contenitori, l'esecuzione di contenitori interrotti e altro ancora. I comandi relativi alla manipolazione delle immagini, o le opzioni interattive dell'utente, non sono rilevanti in Docker Compose perché indirizzano un contenitore. Il file docker-compose.yml viene utilizzato per definire i servizi di un'applicazione e include varie opzioni di configurazione. Ad esempio, l'opzione build definisce opzioni di configurazione come il percorso Dockerfile, l'opzione comando consente di sovrascrivere i comandi Docker predefiniti e altro ancora. La prima versione beta pubblica di Docker Com-pose (versione 0.0.1) è stata rilasciata il 21 dicembre 2013. La prima versione pronta per la produzione (1.0) è stata resa disponibile il 16 ottobre 2014.
\section*{3.2 \hspace{1cm} Linux}
Linux è una famiglia di sistemi operativi Unix open-source basati sul kernel Linux, un kernel del sistema operativo rilasciato per la prima volta il 17 settembre 1991 da Linus Torvalds. Linux è tipicamente impacchettato in una distribuzione Linux. \\

Le distribuzioni includono il kernel Linux e il software di sistema di supporto e le librerie, molte delle quali sono fornite dal progetto GNU. Molte distribuzioni Linux usano la parola "Linux" nel loro nome, ma la Free Software Foundation usa il nome "GNU / Linux" per enfatizzare l'importanza del software GNU, causando alcune controversie.

\section*{3.3 \hspace{1cm} iptables}
iptables è un programma di utilità per lo spazio utente che consente a un amministratore di sistema di configurare le regole del filtro dei pacchetti IP del firewall del kernel Linux, implementate come diversi moduli Netfilter. I filtri sono organizzati in diverse tabelle, che contengono catene di regole su come trattare i pacchetti di traffico di rete. Diversi moduli e programmi del kernel sono attualmente utilizzati per diversi protocolli; iptables si applica a IPv4, ip6tables a IPv6, arptables a ARP ed ebtables a frame Ethernet. \\

iptables richiede privilegi elevati per funzionare e deve essere eseguito dall'utente root, altrimenti non funziona. Sulla maggior parte dei sistemi Linux, iptables è installato come / usr / sbin / iptables e documentato nelle sue pagine man, che possono essere aperte usando man iptables una volta installato. Può anche essere trovato in / sbin / iptables, ma poiché iptables è più simile a un servizio piuttosto che a un "binario essenziale", la posizione preferita rimane / usr / sbin.