% Chapter 1

\chapter{Traccia} % Main chapter title

\label{Chapter1} % For referencing the chapter elsewhere, use \ref{Chapter1} 

%----------------------------------------------------------------------------------------

% Define some commands to keep the formatting separated from the content 
\newcommand{\keyword}[1]{\textbf{#1}}
\newcommand{\tabhead}[1]{\textbf{#1}}
\newcommand{\code}[1]{\texttt{#1}}
\newcommand{\file}[1]{\texttt{\bfseries#1}}
\newcommand{\option}[1]{\texttt{\itshape#1}}

%----------------------------------------------------------------------------------------

La società di distribuzione dell'energia elettrica intende sviluppare un archivio digitale di tutte le linee elettriche che gestisce.
Il sistema deve permettere di mappe sul territorio la collocazione delle cabine di distribuzione e dei tralicci delle linee elettriche.
Il candidato, formulate le opportune ipotesi aggiuntive sulle caratteristiche e la natura del problema in oggetto, sviluppi un'analisi della realtà di
riferimento individuando quali devono essere le specifiche che il sistema deve soddisfare sia dal punto di vista software che dell'infrastruttura di rete.
Sulla base delle specifiche individuate, illustri quali possono essere le soluzioni possibili e scegli a quella chiesto motivato giudizio è la più idonea
a rispondere alle specifiche indicate. \\

Il candidato, quindi, sviluppi l'intero progetto del sistema informatico, in particolare riportando:
lo schema a blocchi dei moduli del prodotto software da realizzare;
il progetto del database completo dello schema concettuale, dello schema logico, delle istruzioni DDL necessarie
per l'implementazione fisica del database e di alcune query necessarie per sviluppare dei moduli software individuati;
il codice di una parte significativa del software in un linguaggio di programmazione a scelta del candidato.